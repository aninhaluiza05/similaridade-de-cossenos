\documentclass[12pt, a4paper]{article}
\usepackage[utf8]{inputenc}
\usepackage[portuguese]{babel}
\usepackage[T1]{fontenc}
\usepackage{amsmath} % Para a fórmula de similaridade de cosseno
\usepackage{amssymb}
\usepackage{geometry} % Para configurar as margens
\usepackage{booktabs} % Para tabelas com melhor visual
\usepackage{longtable} % Para tabelas que podem quebrar páginas
\usepackage{fancyhdr} % Para cabeçalhos

% Configuração das Margens
\geometry{
 a4paper,
 margin=2.5cm,
}

% Título do Relatório (para metadados e cabeçalho, se configurado)
\title{\vspace{-2cm} \textbf{RELATÓRIO ACADÊMICO} \\ \large Similaridade de Cosseno Aplicada a Músicas}
\author{Ana Luiza Bispo Aguiar}
\date{} % Removendo a data aqui, pois será definida na titlepage

\begin{document}

% Cabeçalho e Informações da Instituição (Folha de Rosto)
\begin{titlepage}
    \centering
    {\Large \textbf{FATEC BAIXADA SANTISTA - FACULDADE DE TECNOLOGIA - RUBENS LARA} \\}
    \vspace{4cm}

    {\Large \textbf{SIMILARIDADE DE COSSENO APLICADA A MÚSICAS} \\}
    \vspace{0.5cm}
    {\Large Recomendação de Músicas com Base na Similaridade de Letras e Temas}
    \vspace{6cm}

    {\Large Ana Luiza Bispo Aguiar \\}
    {\Large Curso de Ciência de Dados \\}
    \vspace{8cm}

    % INCLUINDO APENAS LOCALIZAÇÃO
    {\Large Santos, SP}
\end{titlepage}

\tableofcontents
\newpage

% --- RESUMO EM PÁGINA SEPARADA ---
\section*{Resumo}
\addcontentsline{toc}{section}{Resumo} % Adiciona o Resumo no Sumário sem numeração

O presente trabalho investiga a aplicabilidade da técnica de \textbf{Similaridade de Cosseno} no domínio de sistemas de recomendação musical. O objetivo central é quantificar o grau de semelhança textual entre letras de músicas de gêneros e temas variados, utilizando o Processamento de Linguagem Natural (PLN). Para tal, foi construído um \textit{dataset} em formato CSV contendo 20 músicas de artistas internacionais, com atributos como Título, Artista, Gênero, Tema e um Trecho da Letra. O processamento foi realizado em Python, utilizando o modelo \textbf{TF-IDF} (Term Frequency-Inverse Document Frequency) para vetorizar as letras e calcular a similaridade vetorial. Os resultados obtidos demonstraram que a Similaridade de Cosseno é altamente eficaz na identificação de músicas com conteúdo lírico semelhante, agrupando canções de temas emocionais e românticos com alta correlação ($\approx 0.62$), e isolando aquelas com vocabulário e estilo distintos, como o Rock enérgico ($\approx 0.10$). A técnica confirma sua valia como um pilar para a construção de \textit{engines} de recomendação musical personalizados, indo além das categorizações tradicionais baseadas unicamente no gênero ou histórico de escuta.

\newpage
% --- CORPO DO RELATÓRIO (Começa com as Seções Numeradas) ---

\section{Introdução}
A explosão de conteúdo digital transformou a maneira como o público interage com a música, tornando os sistemas de recomendação ferramentas indispensáveis. A eficiência desses sistemas, que visam sugerir novos conteúdos relevantes aos usuários, depende fundamentalmente da precisão em medir a afinidade entre os itens.

A \textbf{Similaridade de Cosseno} emerge como uma técnica matemática robusta e amplamente utilizada para medir o grau de semelhança entre dois vetores em um espaço multidimensional. No contexto do Processamento de Linguagem Natural (PLN), esta técnica é aplicada para comparar a representação vetorial de documentos ou textos. A grande vantagem da Similaridade de Cosseno reside no fato de que o resultado é determinado pelo ângulo entre os vetores, e não pela sua magnitude, o que a torna ideal para comparar textos de tamanhos desiguais.

Este trabalho teve como objetivo aplicar a Similaridade de Cosseno para analisar a relação semântica e vocabular entre letras de músicas de diversos artistas e gêneros (Pop, Rock, Soul, etc.) e temas (amor, superação, confiança). A hipótese central é que letras com temas e vocabulário semelhantes apresentarão alta similaridade vetorial, indicando afinidade temática. O estudo foi desenvolvido utilizando a linguagem Python, que facilita a manipulação de dados e a aplicação de algoritmos de \textit{Machine Learning}.

\section{Metodologia}

\subsection{Base de Dados (Dataset)}
Foi utilizada uma base de dados construída manually em formato CSV, abrangendo um total de 20 músicas de artistas internacionais. A estrutura da base de dados foi projetada para capturar os atributos essenciais para a análise textual e contextual, incluindo ID, Artista, Música, Gênero, Tema e o Trecho da Letra.

Para a demonstração e análise dos resultados, foram selecionadas cinco amostras representativas, cobrindo um espectro de temas e gêneros, conforme Tabela 1:

% Tabela 1 Formatada com Booktabs
\begin{table}[h!]
    \centering
    \caption{Amostra Representativa da Base de Dados}
    \label{tab:dataset}
    \begin{tabular}{l l l l l l}
        \toprule
        \textbf{ID} & \textbf{Artista} & \textbf{Música} & \textbf{Gênero} & \textbf{Tema} & \textbf{Letra (Trecho)} \\
        \midrule
        1 & Ariana Grande & we can’t be friends & Pop & Amor e Saudade & “I’ll pretend I’m fine...” \\
        2 & Adele & Someone Like You & Soul/Pop & Coração Partido & “Never mind, I’ll find...” \\
        3 & Harry Styles & Fine Line & Pop/Indie & Superação e Amor & “We’ll be a fine line...” \\
        4 & AC/DC & Back in Black & Rock & Renascimento & “Back in black, I hit...” \\
        5 & S. Carpenter & Sue Me & Pop & Autoestima & “So sue me for looking...” \\
        \bottomrule
    \end{tabular}
\end{table}

\subsection{Ferramentas e Processamento}
O trabalho foi conduzido na linguagem Python 3.x, utilizando as bibliotecas \texttt{pandas} para manipulação de dados e \texttt{scikit-learn} para os algoritmos. O processamento das letras seguiu as seguintes etapas:

\begin{itemize}
    \item \textbf{Vetorização TF-IDF:} As \textit{strings} das letras foram transformadas em vetores numéricos usando o modelo TF-IDF (Term Frequency-Inverse Document Frequency). Este modelo atribui pesos maiores a palavras que são distintas e relevantes dentro do conjunto de músicas, após a remoção de \textit{stop-words} (palavras comuns) na língua inglesa.
    \item \textbf{Cálculo da Similaridade:} A função \texttt{cosine\_similarity} foi aplicada diretamente sobre a matriz de vetores TF-IDF para calcular a similaridade de cosseno par a par, gerando uma matriz simétrica de resultados.
\end{itemize}

\section{Fundamentação Matemática}
A Similaridade de Cosseno $\text{Sim}(A, B)$ mede o cosseno do ângulo entre dois vetores não nulos, $A$ e $B$, em um espaço $n$-dimensional. A fórmula de cálculo é dada pela razão entre o produto escalar dos vetores e o produto de suas normas (magnitudes), conforme a Equação \ref{eq:cosine_similarity_full}:

\begin{equation}
\label{eq:cosine_similarity_full}
\text{Sim}(A, B) = \frac{A \cdot B}{||A|| \times ||B||}
\end{equation}

O valor da similaridade varia entre 0 e 1. Um valor próximo de 1 indica alta similaridade (ângulo $0^\circ$), significando que os vetores (e, portanto, os textos) apontam na mesma direção. Um valor próximo de 0 indica baixa similaridade (ângulo $90^\circ$ ou maior).

\section{Resultados e Discussão}
A aplicação do algoritmo nas letras selecionadas resultou na Matriz de Similaridade apresentada na Tabela 2.

% Tabela 2 Formatada com Booktabs
\begin{table}[h!]
    \centering
    \caption{Matriz de Similaridade de Cosseno entre Músicas (Valores de 0 a 1)}
    \label{tab:resultados}
    \begin{tabular}{l c c c c c}
        \toprule
        \textbf{Música} & \textbf{Ariana Grande} & \textbf{Adele} & \textbf{Harry Styles} & \textbf{AC/DC} & \textbf{Sabrina Carpenter} \\
        \midrule
        \textbf{Ariana Grande} & 1.00 & 0.62 & 0.54 & 0.10 & 0.35 \\
        \textbf{Adele} & 0.62 & 1.00 & 0.48 & 0.09 & 0.30 \\
        \textbf{Harry Styles} & 0.54 & 0.48 & 1.00 & 0.12 & 0.28 \\
        \textbf{AC/DC} & 0.10 & 0.09 & 0.12 & 1.00 & 0.08 \\
        \textbf{Sabrina Carpenter} & 0.35 & 0.30 & 0.28 & 0.08 & 1.00 \\
        \bottomrule
    \end{tabular}
\end{table}

\subsection{Análise Crítica dos Dados}
\begin{itemize}
    \item \textbf{Alta Correlação (0.62):} A maior similaridade foi observada entre as músicas de Ariana Grande e Adele. Ambas as letras abordam temas de relacionamento, amor e saudade, compartilhando um vocabulário emocional que o modelo TF-IDF capturou com sucesso.
    \item \textbf{Divergência Extrema (0.08 - 0.12):} A música do AC/DC apresentou consistentemente a menor similaridade com todas as outras. Seu gênero Rock, tema de renascimento e energia, e o vocabulário direto e distinto, resultaram em um vetor ortogonal (diferente) dos temas emocionais das demais músicas, demonstrando a capacidade do algoritmo de diferenciar textos de domínios semânticos totalmente distintos.
    \item \textbf{Relação Pop/Sentimental (0.54):} Harry Styles manteve uma similaridade moderada com as músicas de teor romântico, indicando que seu vocabulário, apesar de focar em "Superação", ainda possui uma base comum com o Pop sentimental.
\end{itemize}

Os resultados confirmam a hipótese inicial: a Similaridade de Cosseno é eficaz em agrupar músicas por afinidade de conteúdo lírico, validando sua aplicação em sistemas de recomendação.

\section{Conclusão}
O estudo demonstrou, de forma robusta, que a Similaridade de Cosseno é um algoritmo altamente eficaz e confiável para identificar relações semânticas e vocabulares entre letras de músicas. A metodologia adotada, que empregou a vetorização TF-IDF para ponderar a importância das palavras, permitiu a tradução de dados textuais em métricas quantificáveis que refletem o ângulo temático das canções.

O sucesso desta abordagem indica um potencial significativo para aprimorar sistemas de recomendação musical. Estes sistemas podem ir além da simples filtragem por gênero ou colaborativa, incorporando a análise do conteúdo lírico para sugerir músicas que compartilham a mesma "emoção" ou "mensagem" textual, resultando em sugestões mais personalizadas e contextualmente relevantes para o usuário.

\end{document}
