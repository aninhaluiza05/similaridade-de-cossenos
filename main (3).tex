\documentclass[
    a4paper,
    12pt,
    oneside, % Remove páginas em branco no verso
    openany, % Permite que capítulos comecem em qualquer página
    sumario=tradicional,
    hifeno=false 
]{abntex2}

% --- PACOTES BÁSICOS ---
\usepackage[utf8]{inputenc}
\usepackage[T1]{fontenc}
\usepackage[brazil]{babel}
\usepackage{lmodern}
\usepackage{indentfirst} 
\usepackage{graphicx}
\usepackage{amsmath} % Para fórmulas matemáticas avançadas
\usepackage{amssymb} % Para símbolos matemáticos
\usepackage{booktabs} % Melhora a qualidade das tabelas
\usepackage{url}
\usepackage{listings}
\usepackage{xcolor} 

% --- CONFIGURAÇÕES DE ESTILO ---
\definecolor{codegray}{rgb}{0.5,0.5,0.5}
\definecolor{codepurple}{rgb}{0.58,0,0.82}
\definecolor{backcolour}{rgb}{0.98,0.98,0.98}
\lstdefinestyle{mystyle}{
    backgroundcolor=\color{backcolour},   
    commentstyle=\color{codegray},
    keywordstyle=\color{blue},
    numberstyle=\tiny\color{codegray},
    stringstyle=\color{codepurple},
    basicstyle=\footnotesize\ttfamily,
    breaklines=true,                 
    captionpos=b,                    
    numbers=none,                  
}
\lstset{style=mystyle}

% --- INFORMAÇÕES DO TRABALHO ---
\titulo{Similaridade de Cosseno Aplicada a Músicas: Recomendação de Músicas com Base na Similaridade de Letras e Temas}
\autor{Ana Luiza Bispo Aguiar}
% REMOVENDO A LOCALIZAÇÃO/DATA e colocando apenas o \par para manter o espaçamento vertical
\data{\par} 
\instituicao{%
  \textsc{Fatec Baixada Santista}
  \par
  Curso de Ciência de Dados
}
\tipotrabalho{Relatório Acadêmico}

% --- INÍCIO DO DOCUMENTO ---
\begin{document}

\selectlanguage{brazil}

% Elementos pré-textuais
% \capatitulo % COMENTADO para remover a descrição abaixo dos títulos dos capítulos
\imprimirfolhaderosto % Imprime a Folha de Rosto

% Elementos textuais (Capítulos)
\textual

\chapter{Introdução}
A explosão de conteúdo digital transformou a maneira como o público interage com a música, tornando os sistemas de recomendação ferramentas indispensáveis. A eficiência desses sistemas, que visam sugerir novos conteúdos relevantes aos usuários, depende fundamentalmente da precisão em medir a afinidade entre os itens.

A \textbf{Similaridade de Cosseno} emerge como uma técnica matemática robusta e amplamente utilizada para medir o grau de semelhança entre dois vetores em um espaço multidimensional. No contexto do Processamento de Linguagem Natural (PLN), esta técnica é aplicada para comparar a representação vetorial de documentos ou textos. A grande vantagem da Similaridade de Cosseno reside no fato de que o resultado é determinado pelo ângulo entre os vetores, e não pela sua magnitude, o que a torna ideal para comparar textos de tamanhos desiguais.

Este trabalho teve como objetivo aplicar a Similaridade de Cosseno para analisar a relação semântica e vocabular entre letras de músicas de diversos artistas e gêneros (Pop, Rock, Soul, etc.) e temas (amor, superação, confiança). A hipótese central é que letras com temas e vocabulário semelhantes apresentarão alta similaridade vetorial, indicando afinidade temática. O estudo foi desenvolvido utilizando a linguagem Python, que facilita a manipulação de dados e a aplicação de algoritmos de \textit{Machine Learning}.

\chapter{Metodologia}

\section{Base de Dados (Dataset)}
Foi utilizada uma base de dados construída manualmente em formato CSV, abrangendo um total de 20 músicas de artistas internacionais. A estrutura da base de dados foi projetada para capturar os atributos essenciais para a análise textual e contextual, incluindo ID, Artista, Música, Gênero, Tema e o Trecho da Letra.

Para a demonstração e análise dos resultados, foram selecionadas cinco amostras representativas, cobrindo um espectro de temas e gêneros, conforme Tabela \ref{tab:dataset}:

\begin{table}[h]
    \centering
    \caption{Amostra Representativa da Base de Dados}
    \label{tab:dataset}
    \begin{tabular}{cccccc}
        \toprule
        \textbf{ID} & \textbf{Artista} & \textbf{Música} & \textbf{Gênero} & \textbf{Tema} & \textbf{Letra (Trecho)} \\
        \midrule
        1 & Ariana Grande & we can’t be friends & Pop & Amor e Saudade & “I’ll pretend I’m fine...” \\
        2 & Adele & Someone Like You & Soul/Pop & Coração Partido & “Never mind, I’ll find someone like you...” \\
        3 & Harry Styles & Fine Line & Pop/Indie & Superação e Amor & “We’ll be a fine line between love and pain...” \\
        4 & AC/DC & Back in Black & Rock & Renascimento e Energia & “Back in black, I hit the sack...” \\
        5 & S. Carpenter & Sue Me & Pop & Autoestima e Confiança & “So sue me for looking so pretty tonight...” \\
        \bottomrule
    \end{tabular}
\end{table}

\section{Ferramentas e Processamento}
O trabalho foi conduzido na linguagem Python 3.x, utilizando as bibliotecas \texttt{pandas} para manipulação de dados e \texttt{scikit-learn} para os algoritmos. O processamento das letras seguiu as seguintes etapas:

\begin{itemize}
    \item \textbf{Vetorização TF-IDF}: As \textit{strings} das letras foram transformadas em vetores numéricos usando o modelo TF-IDF (Term Frequency-Inverse Document Frequency).
    \item \textbf{Cálculo da Similaridade}: A função \texttt{cosine\_similarity} foi aplicada diretamente sobre a matriz de vetores TF-IDF para calcular a similaridade de cosseno par a par, gerando uma matriz simétrica de resultados.
\end{itemize}

\chapter{Fundamentação Matemática e Resultados}

\section{Conceito Matemático}
A Similaridade de Cosseno $\text{Sim}(A, B)$ mede o cosseno do ângulo entre dois vetores não nulos, $A$ e $B$, em um espaço $n$-dimensional. A fórmula de cálculo é dada pela razão entre o produto escalar dos vetores e o produto de suas normas (magnitudes), conforme a Equação \ref{eq:similaridade_cosseno}:

\begin{equation}
    \label{eq:similaridade_cosseno}
    \text{Sim}(A, B) = \frac{A \cdot B}{||A|| \times ||B||}
\end{equation}

O valor da similaridade varia entre 0 e 1. Um valor próximo de 1 indica alta similaridade (ângulo $0^\circ$), e um valor próximo de 0 indica baixa similaridade (ângulo $90^\circ$ ou maior).

\section{Análise dos Resultados Obtidos}
A aplicação do algoritmo nas letras selecionadas resultou na Matriz de Similaridade apresentada na Tabela \ref{tab:resultados}.

\begin{table}[h]
    \centering
    \caption{Matriz de Similaridade de Cosseno entre Músicas (Valores de 0 a 1)}
    \label{tab:resultados}
    \begin{tabular}{cccccc}
        \toprule
        \textbf{Música} & \textbf{Ariana Grande} & \textbf{Adele} & \textbf{Harry Styles} & \textbf{AC/DC} & \textbf{Sabrina Carpenter} \\
        \midrule
        \textbf{Ariana Grande} & 1.00 & 0.62 & 0.54 & 0.10 & 0.35 \\
        \textbf{Adele} & 0.62 & 1.00 & 0.48 & 0.09 & 0.30 \\
        \textbf{Harry Styles} & 0.54 & 0.48 & 1.00 & 0.12 & 0.28 \\
        \textbf{AC/DC} & 0.10 & 0.09 & 0.12 & 1.00 & 0.08 \\
        \textbf{Sabrina Carpenter} & 0.35 & 0.30 & 0.28 & 0.08 & 1.00 \\
        \bottomrule
    \end{tabular}
\end{table}

\textbf{Análise Crítica dos Dados:}

\begin{itemize}
    \item \textbf{Alta Correlação ($\approx 0.62$):} A maior similaridade foi observada entre as músicas de Ariana Grande e Adele.
    \item \textbf{Divergência Extrema ($\approx 0.10$):} A música do AC/DC apresentou consistentemente a menor similaridade, validando a capacidade do algoritmo de diferenciar domínios semânticos distintos.
\end{itemize}

Os resultados confirmam a hipótese inicial: a Similaridade de Cosseno é eficaz em agrupar músicas por afinidade de conteúdo lírico, validando sua aplicação em sistemas de recomendação.

\chapter{Conclusão}

O estudo demonstrou, de forma robusta, que a Similaridade de Cosseno é um algoritmo altamente eficaz e confiável para identificar relações semânticas e vocabulares entre letras de músicas. A metodologia adotada permitiu a tradução de dados textuais em métricas quantificáveis que refletem o ângulo temático das canções.

O sucesso desta abordagem indica um potencial significativo para aprimorar sistemas de recomendação musical. Estes sistemas podem ir além da simples filtragem por gênero ou colaborativa, incorporando a análise do conteúdo lírico para sugerir músicas que compartilham a mesma "emoção" ou "mensagem" textual, resultando em sugestões mais personalizadas e contextualmente relevantes para o usuário.

\end{document}
